\chapter*{Introduction}		 		 		 % DO NOT TOUCH!
\addcontentsline{toc}{chapter}{Introduction} % DO NOT TOUCH!

Solving scientific and engineering problems often involves employing various procedures to find solutions. One common approach is to formulate the problem as a system of equations, allowing the use of established methods. In practice, systems of linear equations arise in fields such as engineering, physics, economics, and computer science. On the other hand, systems of partial differential equations are prevalent in fields such as electrodynamics, fluid dynamics, thermodynamics, and diffusion. These systems can often be solved by representing them in matrix form and applying algebraic operations to find the solution.

There are numerous methods available for solving systems of equations, with Gaussian Elimination Method (GEM) and Lower-Upper decomposition with Pivoting (LUP) being among the most well-known. Using LUP to solve systems of equations involves two distinct stages: decomposition and substitution. In both stages, two main groups of numerical methods can be employed: direct or iterative. Direct methods arrive at the result in a finite number of steps, while iterative methods improve an initial estimate until a desired accuracy of the result is reached. However, direct methods are often sequential and difficult to parallelize, which limits their scalability on larger problems. In contrast, iterative methods can be effectively parallelized. Taking advantage of parallelized procedures becomes particularly beneficial for larger problems when utilizing Graphics Processing Units (GPUs), as they are designed to handle substantial concurrent workloads.

The primary objective of this thesis is to introduce and implement an iterative parallel LUP algorithm on the GPU and evaluate its performance. Additionally, this thesis presents a project that aims to facilitate the procedures used in both stages of solving systems of equations with LUP.

The first chapter introduces fundamental theoretical knowledge related to Nvidia GPUs and the application of LUP for solving systems of equations. In the second chapter, the developed project and other functionalities implemented to facilitate the use of the iterative algorithm are presented. Finally, the last chapter provides a comprehensive analysis in the form of a benchmark, evaluating the performance of the procedures available in the project.